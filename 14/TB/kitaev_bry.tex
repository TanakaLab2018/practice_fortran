\documentclass{jsarticle}

\usepackage{amsmath,amssymb}
\usepackage{bm}
\usepackage{braket}
\usepackage[dvipdfmx]{graphicx}
\usepackage{here}

\begin{document}
\part{kitaevモデルハミルトニアンについて}
	\section{定義}
		kitaevモデルハミルトニアン$\mathcal{H}$は以下のように表される。

		\begin{equation}
			\mathcal{H}=\int \vec{\Psi}^\dagger \tilde{H}\vec{\Psi}dr
		\end{equation}

		\begin{equation}
			\int dr=\int dxdy
		\end{equation}

		\begin{equation}
			\vec{\Psi}=
			\begin{bmatrix}
				\Psi_\uparrow \\
				\Psi_\downarrow \\
				\Psi_\uparrow^\dagger \\
				\Psi_\downarrow^\dagger
			\end{bmatrix}
		\end{equation}

		\begin{equation}
			\tilde{H}=
			\begin{bmatrix}
				\hat{h}(r) & \hat{\Delta}(r) \\
				-\hat{\Delta}^\ast(r) & -\hat{h}^\ast(r)
			\end{bmatrix}
		\end{equation}

		\begin{equation}
			\hat{h}=\left[-\frac{\bar{h}^2}{2m}\nabla^2-\mu_F \right]\hat{\sigma_0} \\
			\left( \hat{\sigma_0}:単位行列 \right)
		\end{equation}

		\begin{equation}
			\hat{\Delta}(r)=
			\begin{matrix}
				\Delta_0 \left( i \hat{sigma_2} \right) \\
				\Delta_0\frac{i\partial x}{k_F}\hat{sigma_1} \\
				\Delta_0\frac{1}{k_f} \left( \hat{sigma_1}+i\hat{sigma_2} \right)
			\end{matrix}
		\end{equation}

		\begin{equation}
			\vec{\Psi}=\frac{1}{\sqrt{Ly}}\sum_{ky} \vec{\Psi}_{ky}(x) e^{ik_yy}
		\end{equation}

\end{document}
