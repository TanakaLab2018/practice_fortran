\documentclass{jarticle}
\usepackage{amsmath,amssymb}
\usepackage{amsmath}
\usepackage[dvipdfmx]{graphicx}
\usepackage{here}
\usepackage{pifont}
\usepackage[left=2cm, right=2cm]{geometry}
\setcounter{MaxMatrixCols}{20}
\begin{document}
 (2)前回までに行った計算を$p_{x}-wave$に対して行う。すなわち$\hat{\Delta}(r)$を以下のようにする。
 \begin{align}
 \hat{\Delta}(r)=\Delta_0\dfrac{i\partial x}{k_F}\hat\sigma_1
 \end{align}
 このとき
  \begin{align}
 \hat{\sigma_1}=
 \begin{pmatrix}
 0 & 1\\
 1 & 0
 \end{pmatrix}
 \end{align}
 である。従って
     \begin{align}
 \tilde{H}&=
 \begin{pmatrix}
 (-\dfrac{\hbar^2}{2m}\nabla^2-\mu_F)\begin{pmatrix}
 1 & 0 \\
 0 & 1
 \end{pmatrix} & \dfrac{\Delta_0}{k_{F}} \begin{pmatrix}
 0 & i\partial x \\
 i\partial x & 0
 \end{pmatrix} \\
 -\left(\dfrac{\Delta_0}{k_{F}}\right)^{*}
 \begin{pmatrix}
 0 & -i\partial x \\
 -i\partial x& 0
 \end{pmatrix} & (\dfrac{\hbar^2}{2m}\nabla^2+\mu_F)\begin{pmatrix}
 1 & 0 \\
 0 & 1
 \end{pmatrix}
 \end{pmatrix}
 \\&=\begin{pmatrix}
 \xi & 0 & 0 & \eta \\ 
 0 & \xi & \eta & 0 \\ 
 0 & \eta^{'} & -\xi & 0 \\ 
 \eta^{'} & 0 & 0 & -\xi
 \end{pmatrix} 
 \end{align}
 となる。ここで
 \begin{align}
 -\dfrac{\hbar^2}{2m}\nabla^2-\mu_F&=\xi\\
 \dfrac{\Delta_0}{k_{F}}i\partial x&=\eta\\
 \left(\dfrac{\Delta_0}{k_{F}}\right)^{*}i\partial x&=\eta^{'}
 \end{align}
 とおいた。前回との違いは$\eta$とその中に$x$の微分が入っていることなので、この項を検討していく。例えば
 \begin{align}
 \psi_{\downarrow}e^{-ik_{y}^{'}y}\eta\psi_{\uparrow}e^{ik_{y}^{'}y}
 \end{align}
 は
  \begin{align}
 \psi_{\downarrow}e^{-ik_{y}^{'}y}\eta\psi_{\uparrow}e^{ik_{y}^{'}y}&= \psi_{\downarrow}e^{-ik_{y}^{'}y}\dfrac{\Delta_0}{k_{F}}i\partial x\psi_{\uparrow}e^{ik_{y}^{'}y}\\
 &= \psi_{\downarrow}e^{-ik_{y}^{'}y}\dfrac{\Delta_0}{k_{F}}ie^{ik_{y}^{'}y}\partial x\psi_{\uparrow}
 \end{align}
となる。$\partial x\psi_{\uparrow}$を中心差分を用いて離散化していく。
\begin{align}
\dfrac{\partial}{\partial x}f(x)=\dfrac{f(x+h)-f(x-h)}{2h}
\end{align}
刻み幅$1$で$x$を$i$で書けば
\begin{align}
\dfrac{\partial}{\partial x}f(i)=\dfrac{f(x+i)-f(x-i)}{2}
\end{align}
であるから
これより$\mathcal{H}$は
\begin{align}
\mathcal{H}
=\sum_{i=1}^{L_y+1}\displaystyle\sum_{k_y}\Big[\big\{\dfrac{\hbar^2}{2m}(k_y^2+2)-\mu\big\}\psi_{i\uparrow}^{\dagger}\psi_{i\uparrow}-\Lambda\psi_{i\uparrow}^{\dagger}\psi_{i+1\uparrow}-\Lambda\psi_{i\uparrow}^{\dagger}\psi_{i-1\uparrow}+\dfrac{\Delta_0}{2k_{F}}i\psi_{i\uparrow}^{\dagger}\psi_{{i+1}\uparrow}^{\dagger}-\dfrac{\Delta_0}{2k_{F}}i\psi_{i\uparrow}^{\dagger}\psi_{{i-1}\uparrow}^{\dagger}\cdots\Big]dx
\end{align}
ここで
\begin{align}
\Lambda=\dfrac{\hbar^2}{2m}
\end{align}
とおいた。
\begin{align}
\varepsilon_{k_y}=\dfrac{\hbar^2}{2m}(k_y^2+2)-\mu\\
\dfrac{\Delta_0}{2k_{F}}i=\zeta\\
 \left(\dfrac{\Delta_0}{k_{F}}\right)^{*}i=\zeta^{'}
\end{align}
とすると全体の$\mathcal{H}$は行列で書けて三つ分書くと
\begin{align}
\mathcal{H}=
\begin{bmatrix}
\psi_{1\uparrow}^\dagger \\ 
\psi_{1\downarrow}^\dagger \\ 
\psi_{1\uparrow} \\ 
\psi_{1\downarrow} \\ 
\psi_{2\uparrow}^\dagger \\ 
\psi_{2\downarrow}^\dagger \\ 
\psi_{2\uparrow} \\ 
\psi_{2\downarrow} \\ 
\psi_{3\uparrow}^\dagger \\ 
\psi_{3\downarrow}^\dagger \\ 
\psi_{3\uparrow} \\ 
\psi_{3\downarrow}
\end{bmatrix} 
^T
\begin{bmatrix}
\varepsilon_{k_y} & 0 & 0 & 0 & -\Lambda & 0 & 0 & -\zeta & 0 & 0 & 0 & 0 \\ 
0 & \varepsilon_{k_y} & 0 & 0 & 0 & -\Lambda & -\zeta & 0 & 0 & 0 & 0 & 0 \\ 
0 & 0 & -\varepsilon_{k_y} & 0 & 0 & \zeta^{*} & \Lambda & 0 & 0 & 0 & 0 & 0 \\ 
0 & 0 & 0 & -\varepsilon_{k_y} & \zeta^{*} & 0 & 0 & \Lambda & 0 & 0 & 0 & 0 \\ 
-\Lambda & 0 & 0 & \zeta & \varepsilon_{k_y} & 0 & 0 & 0 & -\Lambda & 0 & 0 & -\zeta \\ 
0 & -\Lambda & \zeta & 0 & 0 & \varepsilon_{k_y} & 0 & 0 & 0 & -\Lambda & -\zeta & 0 \\ 
0 & -\zeta^{*} & \Lambda & 0 & 0 & 0 & -\varepsilon_{k_y} & 0 & 0 & \zeta^{*} & \Lambda & 0 \\ 
-\zeta^{*} & 0 & 0 & \Lambda & 0 & 0 & 0 & -\varepsilon_{k_y} & \zeta^{*} & 0 & 0 & \Lambda \\ 
0 & 0 & 0 & 0 & -\Lambda & 0 & 0 & \zeta & \varepsilon_{k_y} & 0 & 0 & 0 \\ 
0 & 0 & 0 & 0 & 0 & -\Lambda & \zeta & 0 & 0 & \varepsilon_{k_y} & 0 & 0 \\ 
0 & 0 & 0 & 0 & 0 & -\zeta^{*} & \Lambda & 0 & 0 & -0 & -\varepsilon_{k_y} & 0 \\ 
0 & 0 & 0 & 0 & -\zeta^{*} & 0 & 0 & \Lambda & 0 & 0 & 0 & -\varepsilon_{k_y}
\end{bmatrix} 
\begin{bmatrix}
\psi_{1\uparrow} \\ 
\psi_{1\downarrow} \\ 
\psi_{1\uparrow}^\dagger \\ 
\psi_{1\downarrow}^\dagger \\ 
\psi_{2\uparrow} \\ 
\psi_{2\downarrow} \\ 
\psi_{2\uparrow}^\dagger \\ 
\psi_{2\downarrow}^\dagger \\ 
\psi_{3\uparrow} \\ 
\psi_{3\downarrow} \\ 
\psi_{3\uparrow}^\dagger \\ 
\psi_{3\downarrow}^\dagger
\end{bmatrix} 
\end{align}
となる。ここから$\mathcal{H}$を数値計算で対角化していく。
また$k_{F}$は
\begin{align}
\mu=\dfrac{\hbar^{2}k_F^{2}}{2m}
\end{align}
から求める。
\end{document}