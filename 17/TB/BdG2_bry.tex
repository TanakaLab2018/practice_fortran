\documentclass{jsarticle}

\usepackage{amsmath,amssymb}
\usepackage{bm}
\usepackage{braket}
\usepackage[dvipdfmx]{graphicx}
\usepackage{here}
\makeatletter
\c@MaxMatrixCols=12
\makeatother

\begin{document}
\part{BdGモデルハミルトニアンについて}

	\section{問題2}
		\subsection{ハミルトニアン $\mathcal{H}$ をフーリエ変換せよ}
	今度は、$\hat{\Delta}(r)=\Delta_0\frac{i\partial x}{k_F}\hat{\sigma}_1$のときを考える。求める式は、

		\begin{align}
			\mathcal{H}=\int \int \left( \frac{1}{\sqrt{L_y}}\sum_{k_y} \vec{\Psi}^\dagger _{k_y}(x) e^{-ik_yy} \right) \tilde{H} \left( \frac{1}{\sqrt{L_y}}\sum_{k_y'} \vec{\Psi}_{k_y'}(x) e^{ik_y'y} \right) dxdy
			\label{hamil1}
		\end{align}

		ここで

		\begin{align}
			\hat{\sigma}_0=
			\begin{bmatrix}
				1 & 0 \\
				0 & 1
			\end{bmatrix},
			\hat{\sigma}_1=
			\begin{bmatrix}
				0 & 1 \\
				1 & 0
			\end{bmatrix},
			\hat{\sigma}_2=
			\begin{bmatrix}
				0 & -i \\
				i & 0
			\end{bmatrix} ,
			\hat{\sigma}_3=
			\begin{bmatrix}
				1 & 0 \\
				0 & -1
			\end{bmatrix}
		\end{align}

		より、ハミルトニアン$\mathcal{H}$は、

		\begin{align}
			\mathcal{H}=
			\begin{bmatrix}
				-\frac{\hbar^2}{2m}\nabla^2-\mu_F & 0 & 0 & \Delta_0\frac{i\partial x}{k_F} \\
				0 & -\frac{\hbar^2}{2m}\nabla^2-\mu_F & \Delta_0\frac{i\partial x}{k_F} & 0 \\
				0 & \Delta_0^\ast\frac{i\partial x}{k_F} & \frac{\hbar^2}{2m}\nabla^2+\mu_F & 0 \\
				 \Delta_0^\ast\frac{i\partial x}{k_F} & 0 & 0 & \frac{\hbar^2}{2m}\nabla^2+\mu_F
			\end{bmatrix}
		\end{align}

		これを、式\eqref{hamil1}に代入して計算していく。 \\
		このとき、

		(以下、まだ訂正してない)


		\begin{align}
			\vec{\Psi}_{k_y}^\dagger(x) e^{-ik_yy} \tilde{H}  \vec{\Psi}_{k_y'}(x) e^{ik_y'y} =
			e^{-ik_yy} \left[ \Psi_\uparrow^\dagger \left( -\frac{\hbar^2}{2m}\nabla^2-\mu_F \right) +\Psi_\downarrow^\dagger \Delta_0^\ast \right] \Psi_\uparrow e^{ik'_yy} \nonumber\\
			+e^{-ik_yy} \left[ \Psi_\downarrow^\dagger \left( -\frac{\hbar^2}{2m}\nabla^2-\mu_F \right) -\Psi_\uparrow^\dagger \Delta_0^\ast \right] \Psi_\downarrow e^{ik'_yy} \nonumber\\
			+ e^{-ik_yy} \left[ -\Psi_\downarrow\Delta_0 +\Psi_\uparrow \left( \frac{\hbar^2}{2m}\nabla^2+\mu_F \right) \right] \Psi_\uparrow^\dagger e^{ik'_yy} \nonumber\\
			+e^{-ik_yy} \left[ \Psi_\uparrow\Delta_0 +\Psi_\downarrow \left( \frac{\hbar^2}{2m}\nabla^2+\mu_F \right) \right] \Psi_\downarrow^\dagger e^{ik'_yy}
		\end{align}

		また、

		\begin{align}
		\int \left( \sum_{k_y}\sum_{k'_y}\Psi^\dagger(k) e^{-ik_yy}\Psi(k')e^{ik_y'y} \right) dy \nonumber \\
		=L_y\sum_{k_y}\sum_{k'_y}\Psi^\dagger(k) \Psi(k')\delta \left( k-k' \right) \nonumber \\
		=L_y\sum_{k_y}\Psi^\dagger(k)\Psi(k)
		\end{align}

		\begin{align}
			\nabla^2 \left[ \Psi_\uparrow e^{ik'_yy} \right]=
			\frac{\hbar^2}{2m}
			\left( -k_y^2\Psi e^{ik'_yy} + e^{ik'_yy}\frac{\partial^2}{\partial x^2}\Psi \right)
		\end{align}


		から、式\eqref{hamil1}は、

		\begin{align}
			\mathcal{H}=\int dx \sum_{k_y}
			\left[ \Psi_\uparrow^\dagger \left( \frac{\hbar^2k_y^2}{2m}-\mu_F \right)\Psi_\uparrow
			+\Psi_\uparrow^\dagger \left(- \frac{\hbar^2}{2m}\frac{\partial^2}{\partial x^2}\right)\Psi_\uparrow
			+\Psi_\downarrow^\dagger \Delta_0^\ast \Psi_\uparrow \right. \nonumber \\ \left.+
			\Psi_\downarrow^\dagger \left( \frac{\hbar^2k_y^2}{2m}-\mu_F \right)\Psi_\downarrow
			+\Psi_\downarrow^\dagger \left(- \frac{\hbar^2}{2m}\frac{\partial^2}{\partial x^2} \right) \Psi_\downarrow
			-\Psi_\uparrow^\dagger \Delta_0^\ast \Psi_\downarrow \cdots
			\right]
			\label{hamil2}
		\end{align}

		\subsection{差分近似をしよう}
		刻み幅を1として、差分近似をする。微小変化$h$周りにマクローリン展開を行うと、

		\begin{align}
			\Psi\left(x+h\right)=\Psi\left(x\right)+h\frac{\partial}{\partial x}\Psi\left(x\right)+\frac{h^2}{2!}\frac{\partial^2}{\partial x^2}\Psi\left(x\right)
			\label{macro+}
		\end{align}

		\begin{align}
			\Psi\left(x-h\right)=\Psi\left(x\right)-h\frac{\partial}{\partial x}\Psi\left(x\right)+\frac{h^2}{2!}\frac{\partial^2}{\partial x^2}\Psi\left(x\right)
			\label{macro-}
		\end{align}

		よって、差分近似は式\eqref{macro+},式\eqref{macro-}の方程式で求めることができ、刻み幅$h=1$とすると、

		\begin{align}
			\frac{\partial}{\partial x}\Psi\left(x\right)=
			\frac{\Psi\left(x+1\right)-\Psi\left(x-1\right)}{2}
		\end{align}

		\begin{align}
			\frac{\partial^2}{\partial x^2}\Psi\left(x\right)=
			\Psi\left(x+1\right)-2\Psi\left(x\right)+\Psi\left(x-1\right)
			\label{macro2}
		\end{align}

		\subsection{xを離散化せよ}
		式\eqref{macro2}を式\eqref{hamil2}に代入して、行列で表す。このとき、離散化した波動関数を以下の式に定義する。

		\begin{align}
			\vec{\Psi}=
			\begin{bmatrix}
				\Psi_{1\uparrow} \\
				\Psi_{1\downarrow} \\
				\Psi_{1\uparrow}^\dagger \\
				\Psi_{1\downarrow}^\dagger \\
				\Psi_{2\uparrow} \\
				\cdot \\
				\cdot
			\end{bmatrix},
			\Psi_{n\pm 1\uparrow}=\Psi_{n\uparrow}\left(x\pm 1\right)
		\end{align}

		代入した式は、

		\begin{align}
			\mathcal{H}= \sum_{i=1}^N \sum_{k_y}
			\left[ \Psi^\dagger_{i\uparrow} \left( \frac{\hbar^2k_y^2}{2m}-\mu_F \right)\Psi_{i\uparrow}
			+\Psi^\dagger_{i\uparrow} \left( \Psi_{i+1\uparrow}\left(x\right)-2\Psi_{i\uparrow}\left(x\right)+\Psi_{i-1\uparrow}\left(x\right)
			\right)+\Psi^\dagger_{i\downarrow} \Delta_0^\ast \Psi_{i\uparrow} \cdots
			\right]
			\label{hamil3}
		\end{align}

		$N=3$と置くとき、ハミルトニアンを2次形式で表すと、

		\begin{align}
			\mathcal{H}=
			\begin{bmatrix}
				\Psi_{1\uparrow}^\dagger \\
				\Psi_{1\downarrow}^\dagger \\
				\Psi_{1\uparrow} \\
				\Psi_{1\downarrow} \\
				\Psi_{2\uparrow}^\dagger \\
				\Psi_{2\downarrow}^\dagger \\
				\Psi_{2\uparrow} \\
				\Psi_{2\downarrow} \\
				\Psi_{3\uparrow}^\dagger \\
				\Psi_{3\downarrow}^\dagger \\
				\Psi_{3\uparrow} \\
				\Psi_{3\downarrow}
			\end{bmatrix}
			^T
			\begin{bmatrix}
				\varepsilon_{k_y} & 0 & 0 & \Delta_0 & -\varLambda & 0 & 0 & 0 & 0 & 0 & 0 & 0 \\
				0 & \varepsilon_{k_y} & -\Delta_0 & 0 & 0 & -\varLambda & 0 & 0 & 0 & 0 & 0 & 0 \\
				0 & -\Delta_0^\ast & -\varepsilon_{k_y} & 0 & 0 & 0 & \varLambda & 0 & 0 & 0 & 0 & 0 \\
				\Delta_0^\ast & 0 & 0 & -\varepsilon_{k_y} & 0 & 0 & 0 & \varLambda & 0 & 0 & 0 & 0 \\
				-\varLambda & 0 & 0 & 0 & \varepsilon_{k_y} & 0 & 0 & \Delta_0 & -\varLambda & 0 & 0 & 0 \\
				0 & -\varLambda & 0 & 0 & 0 & \varepsilon_{k_y} & -\Delta_0 & 0 & 0 & -\varLambda & 0 & 0 \\
				0 & 0 & \varLambda & 0 & 0 & -\Delta_0^\ast & -\varepsilon_{k_y} & 0 & 0 & 0 & \varLambda & 0 \\
				0 & 0 & 0 & \varLambda & \Delta_0^\ast & 0 & 0 & -\varepsilon_{k_y} & 0 & 0 & 0 & \varLambda \\
				0 & 0 & 0 & 0 & -\varLambda & 0 & 0 & 0 & \varepsilon_{k_y} & 0 & 0 & \Delta_0 \\
				0 & 0 & 0 & 0 & 0 & -\varLambda & 0 & 0 & 0 & \varepsilon_{k_y} & -\Delta_0 & 0 \\
				0 & 0 & 0 & 0 & 0 & 0 & \varLambda & 0 & 0 & -\Delta_0^\ast & -\varepsilon_{k_y} & 0 \\
				0 & 0 & 0 & 0 & 0 & 0 & 0 & \varLambda & -\Delta_0^\ast & 0 & 0 & -\varepsilon_{k_y}
			\end{bmatrix}
			\begin{bmatrix}
				\Psi_{1\uparrow} \\
				\Psi_{1\downarrow} \\
				\Psi_{1\uparrow}^\dagger \\
				\Psi_{1\downarrow}^\dagger \\
				\Psi_{2\uparrow} \\
				\Psi_{2\downarrow} \\
				\Psi_{2\uparrow}^\dagger \\
				\Psi_{2\downarrow}^\dagger \\
				\Psi_{3\uparrow} \\
				\Psi_{3\downarrow} \\
				\Psi_{3\uparrow}^\dagger \\
				\Psi_{3\downarrow}^\dagger \\
			\end{bmatrix}
		\end{align}

		ここで

		\begin{align}
			\varepsilon_{k_y}=\frac{\hbar^2}{2m}(k_y^2+2)-\mu_F
		\end{align}

		\begin{align}
			\varLambda=\frac{\hbar^2}{2m}
		\end{align}

		とおく

		\section{結果}
		$\mathcal{H}$を数値計算で対角化した。
		$N=100$、$k_y$を横軸、固有値を縦軸として分散関係をプロットした。結果は以下のとおりである。


\end{document}
