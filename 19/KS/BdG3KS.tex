\documentclass{jarticle}
\usepackage{amsmath,amssymb}
\usepackage{amsmath}
\usepackage[dvipdfmx]{graphicx}
\usepackage{here}
\usepackage{pifont}
\usepackage[left=0.5cm, right=0.5cm]{geometry}
\setcounter{MaxMatrixCols}{20}
\begin{document}
 (2)前回までに行った計算を$p_{x}-ip_{y}wave$に対して行う。すなわち$\hat{\Delta}(r)$を以下のようにする。
 \begin{align}
 \hat{\Delta}(r)=\Delta_0\dfrac{\partial x+i\partial y}{k_F}\hat\sigma_1
 \end{align}
 このとき
  \begin{align}
 \hat{\sigma_1}=
 \begin{pmatrix}
 0 & 1\\
 1 & 0
 \end{pmatrix}
 \end{align}
 である。従って
     \begin{align}
 \tilde{H}&=
 \begin{pmatrix}
 (-\dfrac{\hbar^2}{2m}\nabla^2-\mu_F)\begin{pmatrix}
 1 & 0 \\
 0 & 1
 \end{pmatrix} & \dfrac{\Delta_0}{k_{F}} \begin{pmatrix}
 0 & \partial x+i\partial y \\
 \partial x+i\partial y & 0
 \end{pmatrix} \\
 -\left(\dfrac{\Delta_0}{k_{F}}\right)^{*}
 \begin{pmatrix}
 0 & \partial x-i\partial y \\
 \partial x-i\partial y& 0
 \end{pmatrix} & (\dfrac{\hbar^2}{2m}\nabla^2+\mu_F)\begin{pmatrix}
 1 & 0 \\
 0 & 1
 \end{pmatrix}
 \end{pmatrix}
 \\&=\begin{pmatrix}
 \xi & 0 & 0 & \eta \\ 
 0 & \xi & \eta & 0 \\ 
 0 & \eta^{'} & -\xi & 0 \\ 
 \eta^{'} & 0 & 0 & -\xi
 \end{pmatrix} 
 \end{align}
 となる。ここで
 \begin{align}
 -\dfrac{\hbar^2}{2m}\nabla^2-\mu_F&=\xi\\
 \dfrac{\Delta_0}{k_{F}}(\partial x+i\partial y)&=\eta\\
 \left(\dfrac{\Delta_0}{k_{F}}\right)^{*}(-\partial x+i\partial y)&=\eta^{'}
 \end{align}
 とおいた。前回との違いは$\eta$とその中に$x$の微分が入っていることなので、この項を検討していく。例えば
 \begin{align}
 \psi_{\uparrow}^{\dagger}e^{-ik_{y}^{'}y}\eta\psi_{\downarrow}^{\dagger}e^{ik_{y}^{'}y}
 \end{align}
 は
  \begin{align}
 \psi_{\uparrow}^{\dagger}e^{-ik_{y}^{'}y}\eta\psi_{\downarrow}^{\dagger}e^{ik_{y}^{'}y}&= \psi_{\uparrow}^{\dagger}e^{-ik_{y}^{'}y}\dfrac{\Delta_0}{k_{F}}(\partial x+i\partial y)\psi_{\downarrow}^{\dagger}e^{ik_{y}^{'}y}\\
 &=\bigl(\psi_{\uparrow}^{\dagger}e^{-ik_{y}^{'}y}e^{ik_{y}^{'}y}\dfrac{\partial}{\partial x}\psi_{\downarrow}^{\dagger}-k_{y}\psi_{\uparrow}^{\dagger}e^{-ik_{y}^{'}y}e^{ik_{y}^{'}y}\psi_{\downarrow}^{\dagger}\bigr)\dfrac{\Delta_0}{k_{F}}
 \end{align}
となる。$\partial x\psi_{\uparrow}$を中心差分を用いて離散化していく。
\begin{align}
\dfrac{\partial}{\partial x}f(x)=\dfrac{f(x+h)-f(x-h)}{2h}
\end{align}
刻み幅$1$で$x$を$i$で書けば
\begin{align}
\dfrac{\partial}{\partial x}f(i)=\dfrac{f(x+i)-f(x-i)}{2}
\end{align}
であるから
これより$\mathcal{H}$の前回からの変更点は例えば
\begin{align}
\mathcal{H}
=\int\sum_{i=1}^{L_y+1}\displaystyle\sum_{k_y}\dfrac{\Delta_0}{k_{F}}\bigl\{\psi_{i\uparrow}^{\dagger}\dfrac{1}{2}\bigl(\psi_{i+1\downarrow}^{\dagger}-\psi_{i-1\downarrow}^{\dagger}\bigr)-k_{y}\psi_{i\uparrow}^{\dagger}\psi_{i\downarrow}^{\dagger}\bigr\}
\end{align}
ここで
\begin{align}
\Lambda=\dfrac{\hbar^2}{2m}\\
\varepsilon_{k_y}=\dfrac{\hbar^2}{2m}(k_y^2+2)-\mu\\
\dfrac{\Delta_0}{2k_{F}}=C\\
\end{align}
とすると全体の$\mathcal{H}$は行列で書けて三つ分書くと
\begin{align}
\mathcal{H}=
\begin{bmatrix}
\psi_{1\uparrow}^\dagger \\ 
\psi_{1\downarrow}^\dagger \\ 
\psi_{1\uparrow} \\ 
\psi_{1\downarrow} \\ 
\psi_{2\uparrow}^\dagger \\ 
\psi_{2\downarrow}^\dagger \\ 
\psi_{2\uparrow} \\ 
\psi_{2\downarrow} \\ 
\psi_{3\uparrow}^\dagger \\ 
\psi_{3\downarrow}^\dagger \\ 
\psi_{3\uparrow} \\ 
\psi_{3\downarrow}
\end{bmatrix} 
^T
\begin{bmatrix}
\varepsilon_{k_y} & 0 & 0 & -2Ck_{y} & -\Lambda & 0 & 0 & C & 0 & 0 & 0 & 0 \\ 
0 & \varepsilon_{k_y} & -2Ck_{y} & 0 & 0 & -\Lambda & C & 0 & 0 & 0 & 0 & 0 \\ 
0 & -2Ck_{y}^{*} & -\varepsilon_{k_y} & 0 & 0 & -C^{*} & \Lambda & 0 & 0 & 0 & 0 & 0 \\ 
-2Ck_{y}^{*} & 0 & 0 & -\varepsilon_{k_y} & -C^{*} & 0 & 0 & \Lambda & 0 & 0 & 0 & 0 \\ 
-\Lambda & 0 & 0 & -C & \varepsilon_{k_y} & 0 & 0 & -2Ck_{y} & -\Lambda & 0 & 0 & C \\ 
0 & -\Lambda & -C & 0 & 0 & \varepsilon_{k_y} & -2Ck_{y} & 0 & 0 & -\Lambda & C & 0 \\ 
0 & C^{*} & \Lambda & 0 & 0 & -2Ck_{y}^{*} & -\varepsilon_{k_y} & 0 & 0 & -C^{*} & \Lambda & 0 \\ 
C^{*} & 0 & 0 & \Lambda & -2Ck_{y}^{*} & 0 & 0 & -\varepsilon_{k_y} & -C^{*} & 0 & 0 & \Lambda \\ 
0 & 0 & 0 & 0 & -\Lambda & 0 & 0 & -C & \varepsilon_{k_y} & 0 & 0 & -2Ck_{y} \\ 
0 & 0 & 0 & 0 & 0 & -\Lambda & -C & 0 & 0 & \varepsilon_{k_y} & -2Ck_{y} & 0 \\ 
0 & 0 & 0 & 0 & 0 & C^{*} & \Lambda & 0 & 0 & -2Ck_{y}^{*} & -\varepsilon_{k_y} & 0 \\ 
0 & 0 & 0 & 0 & C^{*} & 0 & 0 & \Lambda & -2Ck_{y}^{*} & 0 & 0 & -\varepsilon_{k_y}
\end{bmatrix} 
\begin{bmatrix}
\psi_{1\uparrow} \\ 
\psi_{1\downarrow} \\ 
\psi_{1\uparrow}^\dagger \\ 
\psi_{1\downarrow}^\dagger \\ 
\psi_{2\uparrow} \\ 
\psi_{2\downarrow} \\ 
\psi_{2\uparrow}^\dagger \\ 
\psi_{2\downarrow}^\dagger \\ 
\psi_{3\uparrow} \\ 
\psi_{3\downarrow} \\ 
\psi_{3\uparrow}^\dagger \\ 
\psi_{3\downarrow}^\dagger
\end{bmatrix} 
\end{align}
となる。ここから$\mathcal{H}$を数値計算で対角化していく。
また$k_{F}$は
\begin{align}
\mu=\dfrac{\hbar^{2}k_F^{2}}{2m}
\end{align}
から求める。
\end{document}