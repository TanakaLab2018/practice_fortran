\documentclass{jarticle}
\usepackage{amsmath,amssymb}
\usepackage{amsmath}
\usepackage[dvipdfmx]{graphicx}
\usepackage{here}
\usepackage{pifont}
\usepackage[left=2cm, right=2cm]{geometry}
\setcounter{MaxMatrixCols}{20}
\begin{document}
	    \begin{align}
	   \mathcal{H}=\int\vec{\Psi}^\dagger(x,y)\hat{H}\vec{\Psi}(x,y)dr
	   \end{align}
	   に
	  \begin{align}
	   \vec{\Psi}=\dfrac{1}{\sqrt{L_y}}\displaystyle\sum_{k_y}\vec{\Psi}_{k_y}(x)e^{ik_yy}
	  \end{align}
	  を代入していく。
	  ここで
	  \begin{align}
	  \vec{\Psi}^\dagger=\dfrac{1}{\sqrt{L_y}}\displaystyle\sum_{k_y^{'}}\vec{\Psi}^\dagger_{k_y^{'}}(x)e^{-ik_y^{'}y}
	  \end{align}
	  であるから
	   \begin{align}
	  \mathcal{H}&=\int\dfrac{1}{\sqrt{L_y}}\displaystyle\sum_{k_y^{'}}\vec{\Psi}^\dagger_{k_y}(x)e^{-ik_yy}\hat{H}\dfrac{1}{\sqrt{L_y}}\displaystyle\sum_{k_y}\vec{\Psi}_{k_y}(x)e^{ik_yy}(x,y)dr\\&=\dfrac{1}{L_y}\underline{\int\displaystyle\sum_{k_y^{'}}\vec{\Psi}^\dagger_{k_y}(x)e^{-ik_yy}\hat{H}\displaystyle\sum_{k_y}\vec{\Psi}_{k_y}(x)e^{ik_yy}(x,y)dr}\\
	  \label{star}
	  \end{align}
	   \\となる。下線部について計算していく。
  \begin{align}
  \tilde{H}&=
  \begin{pmatrix}
  \hat{h}(r) & \hat{\Delta}(r) \\
  -\hat{\Delta}^{*}(r) & -\hat{h}^{*}(r)
  \end{pmatrix}
   \\&=
     \begin{pmatrix}
   (-\dfrac{\hbar^2}{2m}\nabla^2-\mu_F)\hat{\sigma}_0 & \Delta_0(i\sigma_2) \\
   -\Delta^{*}_0(\sigma^{*}_2) & (\dfrac{\hbar^2}{2m}\nabla^2+\mu_F)\hat{\sigma}^{*}_0
   \end{pmatrix}
  \end{align}
  ここで
  $\hat\sigma_0$、$\hat\sigma_2$はパウリ行列であり、
  \begin{align}
  \hat{\sigma_0}=
  \begin{pmatrix}
  1 & 0 \\
  0 & 1
  \end{pmatrix},
  \hat{\sigma_2}=
  \begin{pmatrix}
  0 & -i \\
  i & 0
  \end{pmatrix}
  \end{align}
  である。
    \begin{align}
  \tilde{H}&=
       \begin{pmatrix}
  (-\dfrac{\hbar^2}{2m}\nabla^2-\mu_F)\begin{pmatrix}
  1 & 0 \\
  0 & 1
  \end{pmatrix} & \Delta_0i \begin{pmatrix}
  0 & -i \\
  i & 0
  \end{pmatrix} \\
  -\Delta^{*}_0\begin{pmatrix}
  0 & -1 \\
  1& 0
  \end{pmatrix} & (\dfrac{\hbar^2}{2m}\nabla^2+\mu_F)\begin{pmatrix}
  1 & 0 \\
  0 & 1
  \end{pmatrix}
  \end{pmatrix}
  \\&=\begin{pmatrix}
  \xi & 0 & 0 & \Delta_0 \\ 
  0 & \xi & -\Delta_0 & 0 \\ 
  0 & -\Delta^{*}_0 & -\xi & 0 \\ 
  \Delta^{*}_0 & 0 & 0 & -\xi
  \end{pmatrix} 
  \end{align}
  ここで
  \begin{align}
  -\dfrac{\hbar^2}{2m}\nabla^2-\mu_F=\xi
  \end{align}
  とした。式\ref{star}の下線部は\\
  \begin{align}
  \begin{split}
  	\int&\displaystyle\sum_{k_y}\sum_{k_y^{'}}\begin{pmatrix}
  	\psi_{\uparrow}^{\dagger}e^{-ik_y^{'}y} & \psi_{\downarrow}^{\dagger}e^{-ik_y^{'}y} & \psi_{\uparrow}e^{-ik_y^{'}y} & \psi_{\downarrow}e^{-ik_y^{'}y}
  	\end{pmatrix}
  	 \begin{pmatrix}
  	\xi & 0 & 0 & \Delta_0 \\ 
  	0 & \xi & -\Delta_0 & 0 \\ 
  	0 & -\Delta^{*}_0 & -\xi & 0 \\ 
  	\Delta^{*}_0 & 0 & 0 & -\xi
  	\end{pmatrix}
  	\displaystyle\begin{bmatrix}
  	\psi_{\uparrow}e^{ik_yy} \\
  	\psi_{\downarrow}e^{ik_yy} \\
  	\psi_{\uparrow}^{\dagger} e^{ik_yy}\\
  	\psi_{\downarrow}^{\dagger}e^{ik_yy}
  	\end{bmatrix}dxdy
  	\\=\int\displaystyle\sum_{k_y}&\sum_{k_y^{'}}\psi_{\uparrow}^{\dagger}e^{-ik_y^{'}y}(\xi\psi_{\uparrow}e^{ik_yy}+\Delta^{*}_0\psi_{\downarrow}^{\dagger}e^{ik_yy})\\&
  	                                                                 +\psi_{\downarrow}^{\dagger}e^{-ik_y^{'}y}(\xi\psi_{\downarrow}e^{ik_yy}-\Delta^{*}_0\psi_{\uparrow}^{\dagger}e^{ik_yy})\\&
  	                                                                +\psi_{\uparrow}e^{-ik_y^{'}y}(-\Delta_0\psi_{\downarrow}^{*}e^{ik_yy}-\nabla^{'}\psi_{\uparrow}^{*}e^{ik_yy})\\&
  	                                                                 +\psi_{\downarrow}e^{-ik_y^{'}y}(\Delta_0\psi_{\uparrow}^{*}e^{ik_yy}-\nabla^{'}\psi_{\downarrow}^{*}e^{ik_yy})
  \end{split}	                                                          
\end{align}
第一項について
\begin{align}
&\int\displaystyle\sum_{k_y}\sum_{k_y^{'}}\psi_{\uparrow}^{\dagger}e^{-ik_y^{'}y}(\xi\psi_{\uparrow}e^{ik_yy}+\Delta^{*}_0\psi_{\downarrow}^{\dagger}e^{ik_yy})dxdy\\
&=\int\displaystyle\sum_{k_y}\sum_{k_y^{'}}\psi_{\uparrow}^{\dagger}e^{-ik_y^{'}y}[(-\dfrac{\hbar^2}{2m}\nabla^2-\mu_F)\psi_{\uparrow}e^{ik_yy}+\Delta^{*}_0\psi_{\downarrow}^{\dagger}e^{ik_yy}]dxdy\\
\end{align}
$\nabla^2$の計算を行う。
\begin{align}
\nabla^{2}\psi_{\uparrow}e^{ik_yy}&=(\partial x^2+\partial y^2)\psi_{\uparrow}e^{ik_yy}\\
                                                      &=\partial x^2\psi_{\uparrow}e^{ik_yy}+\partial y^2\psi_{\uparrow}e^{ik_yy}\\
                                                    &=e^{ik_yy}\partial x^2\psi_{\uparrow}-k^{2}_ye^{ik_yy}\psi_{\uparrow}
\end{align}
であるから
\begin{align}
\int\displaystyle\sum_{k_y}\sum_{k_y^{'}}\psi_{\uparrow}^{\dagger}e^{-ik_y^{'}y}\Big[\big\{-\dfrac{\hbar^2}{2m}(\dfrac{\partial^{2}}{\partial x^2}-k^{2}_y)-\mu_F\big\}\psi_{\uparrow}e^{ik_yy}+\Delta^{*}_0\psi_{\downarrow}^{\dagger}e^{ik_yy}\Big]dxdy
\end{align}
ここで
\begin{align}
\int\displaystyle\sum_{k_y}\sum_{k_y^{'}}\psi^{\dagger}e^{-ik_y^{'}y}e^{ik_yy}{\psi}dy=L_y\sum_{k_y}\psi^{\dagger}\psi
\end{align}
を用いて
\begin{align}
\int\displaystyle\sum_{k_y}L_y\Big[\psi_{\uparrow}^{\dagger}\big\{-\dfrac{\hbar^2}{2m}(\dfrac{\partial^{2}}{\partial x^2}-k^{2}_y)-\mu_F\big\}\psi_{\uparrow}+\psi_{\uparrow}^{\dagger}\Delta^{*}_0\psi_{\downarrow}^{\dagger}\Big]dx
\end{align}
従って$\mathcal{H}$は
\begin{align}
\mathcal{H}=\int\displaystyle\sum_{k_y}&\Big[\psi_{\uparrow}^{\dagger}\big\{-\dfrac{\hbar^2}{2m}(\dfrac{\partial^{2}}{\partial x^2}-k^{2}_y)-\mu_F\big\}\psi_{\uparrow}+\psi_{\uparrow}^{\dagger}\Delta^{*}_0\psi_{\downarrow}^{\dagger}\\
&+\psi_{\downarrow}^{\dagger}\big\{-\dfrac{\hbar^2}{2m}(\dfrac{\partial^{2}}{\partial x^2}-k^{2}_y)-\mu_F\big\}\psi_{\downarrow}-\psi_{\downarrow}^{\dagger}\Delta^{*}_0\psi_{\uparrow}^{\dagger}\\
&-\psi_{\uparrow}^{\dagger}\big\{-\dfrac{\hbar^2}{2m}(\dfrac{\partial^{2}}{\partial x^2}-k^{2}_y)-\mu_F\big\}\psi_{\uparrow}^{\dagger}-\psi_{\uparrow}\Delta_0\psi_{\downarrow}^{\dagger}\\
&-\psi_{\downarrow}\big\{-\dfrac{\hbar^2}{2m}(\dfrac{\partial^{2}}{\partial x^2}-k^{2}_y)-\mu_F\big\}\psi_{\downarrow}^{\dagger}+\psi_{\downarrow}\Delta_0\psi_{\uparrow}\Big]dx
\end{align}
積分$\int$を和記号$\sum$にして$x$を離散化していく。刻み幅を$1$とすると
\begin{align}
\int dx\rightarrow\sum_{i=1}^{L_y+1}\Delta x=\sum_{i=1}^{L_y+1}
\end{align}
となる。また波動関数を
\begin{align}
\psi(x)\rightarrow\psi_i
\end{align}
と離散化する。すると$\mathcal{H}$は
\begin{align}
\mathcal{H}&=\sum_{i=1}^{L_y+1}\displaystyle\sum_{k_y}\Big[\psi_{i\uparrow}^{\dagger}\big\{-\dfrac{\hbar^2}{2m}(\dfrac{\partial^{2}}{\partial x^2}-k^{2}_y)-\mu_F\big\}\psi_{i\uparrow}+\psi_{i\uparrow}^{\dagger}\Delta^{*}_0\psi_{i\downarrow}^{\dagger}\cdots\Big]dx\\
&=\sum_{i=1}^{L_y+1}\displaystyle\sum_{k_y}\Big[-\dfrac{\hbar^2}{2m}(\psi_{i\uparrow}^{\dagger}\dfrac{\partial^{2}}{\partial x^2}\psi_{i\uparrow})+(\dfrac{\hbar^2k^{2}_y}{2m}-\mu_F)\psi_{i\uparrow}^{\dagger}\psi_{i\uparrow}+\psi_{i\uparrow}^{\dagger}\Delta^{*}_0\psi_{i\downarrow}^{\dagger}\cdots\Big]dx
\end{align}
ここで$\dfrac{\partial^{2}}{\partial x^2}\psi_{i\uparrow}$を差分近似する。関数を次のように近似する。
\begin{align}
\dfrac{\partial^{2}}{\partial x^2}f(x)=\dfrac{f(x+h)-2f(x)+f(x-h)}{h^2}
\end{align}
刻み幅$1$で$x$を$i$で書けば
\begin{align}
\dfrac{\partial^{2}}{\partial x^2}f(i)=f(i+1)-2f(i)+f(i-1)
\end{align}
これより$\mathcal{H}$は
\begin{align}
\mathcal{H}
=\sum_{i=1}^{L_y+1}\displaystyle\sum_{k_y}\Big[\big\{\dfrac{\hbar^2}{2m}(k_y^2+2)-\mu\big\}\psi_{i\uparrow}^{\dagger}\psi_{i\uparrow}-\Lambda\psi_{i\uparrow}^{\dagger}\psi_{i+1\uparrow}-\Lambda\psi_{i\uparrow}^{\dagger}\psi_{i-1\uparrow}+\psi_{i\uparrow}^{\dagger}\Delta^{*}_0\psi_{i\downarrow}^{\dagger}\cdots\Big]dx
\end{align}
ここで
\begin{align}
\Lambda=\dfrac{\hbar^2}{2m}
\end{align}
とおいた。
\begin{align}
\varepsilon_{k_y}=\dfrac{\hbar^2}{2m}(k_y^2+2)-\mu
\end{align}
とすると全体の$\mathcal{H}$は行列で書けて三つ分書くと
\begin{align}
\mathcal{H}=
\begin{bmatrix}
\psi_{1\uparrow}^\dagger \\ 
\psi_{1\downarrow}^\dagger \\ 
\psi_{1\uparrow} \\ 
\psi_{1\downarrow} \\ 
\psi_{2\uparrow}^\dagger \\ 
\psi_{2\downarrow}^\dagger \\ 
\psi_{2\uparrow} \\ 
\psi_{2\downarrow} \\ 
\psi_{3\uparrow}^\dagger \\ 
\psi_{3\downarrow}^\dagger \\ 
\psi_{3\uparrow} \\ 
\psi_{3\downarrow}
\end{bmatrix} 
^T
 \begin{bmatrix}
	\varepsilon_{k_y} & 0 & 0 & \Delta_0 & -\Lambda & 0 & 0 & 0 & 0 & 0 & 0 & 0 \\ 
	0 & \varepsilon_{k_y} & -\Delta_0 & 0 & 0 & -\Lambda & 0 & 0 & 0 & 0 & 0 & 0 \\ 
	0 & -\Delta_0^{*} & -\varepsilon_{k_y} & 0 & 0 & 0 & \Lambda & 0 & 0 & 0 & 0 & 0 \\ 
	\Delta_0^{*} & 0 & 0 & -\varepsilon_{k_y} & 0 & 0 & 0 & \Lambda & 0 & 0 & 0 & 0 \\ 
	-\Lambda & 0 & 0 & 0 & \varepsilon_{k_y} & 0 & 0 & \Delta_0 & -\Lambda & 0 & 0 & 0 \\ 
	0 & -\Lambda & 0 & 0 & 0 & \varepsilon_{k_y} & -\Delta_0 & 0 & 0 & -\Lambda & 0 & 0 \\ 
	0 & 0 & \Lambda & 0 & 0 & -\Delta_0^{*} & -\varepsilon_{k_y} & 0 & 0 & 0 & \Lambda & 0 \\ 
	0 & 0 & 0 & \Lambda & \Delta_0^{*} & 0 & 0 & -\varepsilon_{k_y} & 0 & 0 & 0 & \Lambda \\ 
	0 & 0 & 0 & 0 & -\Lambda & 0 & 0 & 0 & \varepsilon_{k_y} & 0 & 0 & \Delta_0 \\ 
	0 & 0 & 0 & 0 & 0 & -\Lambda & 0 & 0 & 0 & \varepsilon_{k_y} & -\Delta_0 & 0 \\ 
	0 & 0 & 0 & 0 & 0 & 0 & \Lambda & 0 & 0 & -\Delta_0^{*} & \varepsilon_{k_y} & 0 \\ 
	0 & 0 & 0 & 0 & 0 & 0 & 0 & \Lambda & \Delta_0^{*} & 0 & 0 & -\varepsilon_{k_y}
 \end{bmatrix} 
 \begin{bmatrix}
 \psi_{1\uparrow} \\ 
 \psi_{1\downarrow} \\ 
 \psi_{1\uparrow}^\dagger \\ 
 \psi_{1\downarrow}^\dagger \\ 
 \psi_{2\uparrow} \\ 
 \psi_{2\downarrow} \\ 
 \psi_{2\uparrow}^\dagger \\ 
 \psi_{2\downarrow}^\dagger \\ 
 \psi_{3\uparrow} \\ 
 \psi_{3\downarrow} \\ 
 \psi_{3\uparrow}^\dagger \\ 
 \psi_{3\downarrow}^\dagger
 \end{bmatrix} 
\end{align}
となる。ここから$\mathcal{H}$を数値計算で対角化していく。
\end{document}